\documentclass{article} %文章类型
\usepackage{CJKutf8} %China Japan Korea 
\usepackage{draftwatermark} %水印
\usepackage{graphicx} % 插图
\usepackage{listings} %代码块
\usepackage{xcolor} %颜色
\usepackage{url}
\usepackage{hyperref} %参考引用

%-----------------参考 引用设置--------------
\hypersetup{
	colorlinks=true,
	linkcolor=black,
	%filecolor=blue,
	urlcolor=blue,
	%citecolor=cyan,
}

%--------------------代码块配置--------------
\lstset{
	columns=fixed,       
	%numbers=left,                                        % 在左侧显示行号
	frame=none,                                          % 不显示背景边框
	backgroundcolor=\color[RGB]{245,245,244},            % 设定背景颜色
	keywordstyle=\color[RGB]{40,40,255},                 % 设定关键字颜色
	numberstyle=\footnotesize\color{darkgray},           % 设定行号格式
	commentstyle=\it\color[RGB]{0,96,96},                % 设置代码注释的格式
	stringstyle=\rmfamily\slshape\color[RGB]{128,0,0},   % 设置字符串格式
	showstringspaces=false,                              % 不显示字符串中的空格
	language=C,                                          % 设置语言
}

%--------------------add watermark ---------
\SetWatermarkText{Excelfore}
\SetWatermarkLightness{0.9}
\SetWatermarkScale{6}

%---------------------页眉 页脚 ------------
\usepackage{fancyhdr}
\pagestyle{fancy}
\fancyhf{} 
%\fancyhead[R]{\leftmark} %一级标题
%\fancyhead[R]{\rightmark} %二级标题
%\fancyhead[C]{zhong}
\fancyhead[R]{\scriptsize 3155 Kearney Street, Fremont, CA 94538\\+1.510.868.2500 $\cdot$ \underline{\url{www.excelfore.com}}}
\fancyhead[L]{\includegraphics[scale=2]{excelfore.png}}
%\fancyfoot[C]{centfoot}
\fancyfoot[R]{\thepage}
\fancyfoot[L]{Excelfore}
\renewcommand{\headrulewidth}{1pt} %页眉横线
\renewcommand{\footrulewidth}{0.5pt} %页脚横线


\title{Latex如何生成PDF}
\author{Excelfore 刘小龙}
\begin{document}
	\begin{CJK}{UTF8}{gbsn}

	\maketitle
	
	\newpage
	\tableofcontents %目录

	\newpage % 新建页面让目录独立成页
	
	\section{工具安装}
	\begin{lstlisting}
sudo apt-get install latex-cjk-*
sudo apt install doxygen-latex
	\end{lstlisting}

	\section{修改tex文件}
	进入doxygen生成的latex目录里面
	修改refman.tex文件,可以配置格式

	\section{添加水印}
	\begin{verbatim}
		\usepackage{draftwatermark}

		%--------------------add watermark ---------
		\SetWatermarkText{Excelfore}  %要加水印字符
		\SetWatermarkLightness{0.8}   %透明度
		\SetWatermarkScale{5}         %大小
	\end{verbatim}

	\section{生成PDF}
		\subsection{Doxygen模式生成PDF}
		如果refman.tex是由Doxygen生成的,那么同时也会生成一个Makefile文件。
		我们只需要在latex目录执行 make 命令就可以了
		\begin{lstlisting}
make
		\end{lstlisting}
		\subsection{单独tex文件生成PDF}
		\begin{lstlisting}
xelatex test.tex
pdflatex test.tex
		\end{lstlisting}
	
	\section{常见问题处理}
	\subsection{中文丢字}
	%插入latex语句
	\begin{verbatim}
		\usepackage{CJKutf8}
		\begin{document}                   
		\begin{CJK}{UTF8}{gbsn}
			
		%文档内容
		...
			
		\end{CJK}                                                                  
		\end{document}
	\end{verbatim}


	\newpage
    
\end{CJK}
\end{document}